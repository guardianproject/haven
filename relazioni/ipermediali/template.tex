\usepackage{fancyhdr}
\usepackage{lastpage}
\usepackage[utf8x]{inputenc}
\usepackage[italian]{babel}
\usepackage[colorlinks=true,linkcolor=black]{hyperref}
\usepackage{hhline}
\usepackage{hyperref}
\usepackage[font=small,format=plain,labelfont=bf,up,textfont=it,up]{caption}
\usepackage{indentfirst}
\usepackage[Lenny]{fncychap}
\usepackage{listings}
\usepackage{verbatim}
\usepackage{graphics}
\usepackage{graphicx}
\usepackage{amsmath}
\usepackage{chngpage}
\usepackage{booktabs}
\usepackage{tikz}
\usetikzlibrary{arrows}
\usepackage[hang,centerlast]{subfigure}
\usepackage{color}
\usepackage{caption}
\definecolor{dkgreen}{rgb}{0,0.6,0}
\definecolor{gray}{rgb}{0.5,0.5,0.5}
\definecolor{mauve}{rgb}{0.58,0,0.82}
\DeclareCaptionFont{white}{\color{white}}
\DeclareCaptionFormat{listing}{\colorbox{gray}{\parbox{\linewidth}{#1#2#3}}}
\captionsetup[lstlisting]{format=listing,labelfont=white,textfont=white}
\usepackage{listings}
\lstset{ %
  basicstyle=\footnotesize,           % the size of the fonts that are used for the code
  numbers=left,                   % where to put the line-numbers
  numberstyle=\tiny\color{gray},  % the style that is used for the line-numbers
  stepnumber=1,                   % the step between two line-numbers. If it's 1, each line 
                                  % will be numbered
  numbersep=5pt,                  % how far the line-numbers are from the code
  backgroundcolor=\color{white},      % choose the background color. You must add \usepackage{color}
  showspaces=false,               % show spaces adding particular underscores
  showstringspaces=false,         % underline spaces within strings
  showtabs=false,                 % show tabs within strings adding particular underscores
  %frame=single,                   % adds a frame around the code
  rulecolor=\color{black},        % if not set, the frame-color may be changed on line-breaks within not-black text (e.g. commens (green here))
  tabsize=2,                      % sets default tabsize to 2 spaces
  captionpos=t,                   % sets the caption-position to bottom
  breaklines=true,                % sets automatic line breaking
  breakatwhitespace=false,        % sets if automatic breaks should only happen at whitespace
  title=\lstname,                   % show the filename of files included with \lstinputlisting;
                                  % also try caption instead of title
  keywordstyle=\color{blue},          % keyword style
  commentstyle=\color{dkgreen},       % comment style
  stringstyle=\color{mauve},         % string literal style
  escapeinside={\%*}{*)},            % if you want to add a comment within your code
  morekeywords={*,...}               % if you want to add more keywords to the set
}


\newcommand{\makeTitlePage} {
\begin{titlepage}
\begin{center}
\textsc{università degli studi di Padova}\\
\textsc{corso di laurea magistrale in Informatica}\\
\textsc{anno accademico 2011 - 2012}\\[3cm]

\begin{tikzpicture}
\foreach \i in {0,10,...,100} {
  \foreach \j in {0,30,...,330} {
    \draw[fill=yellow!\i!red] (\j+.01*\i*\i:.04*\i+.001*\i) circle (.01*\i+.1);
  }
}
\foreach \j in {0,30,...,330} {
  \draw[fill=red!70!green] (\j+1000:4.3) circle (.1cm);
}
\end{tikzpicture}

\end{center}
\end{titlepage}}

\newcommand{\upperborder}{
\begin{tikzpicture}
\clip (0,0) rectangle (24.5, 4);
\foreach \i in {0,1,...,6} {
  \fill[red!40!yellow] (\i*4.5, 0) circle (2.5cm);
  \foreach \j in {0,30,...,180} {
    \fill[red!40!yellow] (\i*4.5,0) +(\j:2.5cm) circle (.7cm);
    \foreach \z in {0,40,...,360} {
      \fill[red!40!yellow] (\i*4.5,0) ++(\j:2.5cm) +(\z:1cm) circle (.2cm);
    }
  }
}

\end{tikzpicture}}

\fancypagestyle{testo}{
\pagenumbering{arabic}
\setcounter{page}{1}
%\lhead{\begin{picture}(0,0)(145,4)\upperborder\end{picture}}
\chead{}
\rhead{}
\lfoot{\footnotesize \textsc{Sistemi Ipermediali}}
\cfoot{\thepage}
\rfoot{}
\renewcommand{\headrulewidth}{0.4pt}
\renewcommand{\footrulewidth}{0.4pt}
}
